\documentclass{beamer}
\mode<presentation> {
%\usetheme{Madrid}
%\usetheme{default}
\usepackage{color}
\definecolor{bottomcolour}{rgb}{0.21,0.11,0.21}
\definecolor{middlecolour}{rgb}{0.21,0.11,0.21}
\setbeamercolor{structure}{fg=white}
\setbeamertemplate{frametitle}[default]%[center]
\setbeamercolor{normal text}{bg=black, fg=white}
\setbeamertemplate{background canvas}[vertical shading]
[bottom=bottomcolour, middle=middlecolour, top=black]
\setbeamertemplate{items}[circle]
\setbeamertemplate{navigation symbols}{} %no nav symbols
\setbeamercolor{block title}{use=structure,fg=white,bg=structure.fg!50!red!50!blue!100!green}
\setbeamercolor{block body}{parent=normal text,use=block title,bg=block title.bg!5!white!10!bg,fg=white}
\setbeamertemplate{navigation symbols}{}
}

\usepackage{graphicx} 
\usepackage{booktabs} 
\usepackage[utf8]{inputenc}  
\usepackage[T1]{fontenc}  
\usepackage{geometry}     
\usepackage[francais]{babel} 
\usepackage{eurosym}
\usepackage{verbatim}
\usepackage{ragged2e}
\justifying

\input{cc_beamer}

\title[Les Chiffrofêtes]{Les Chiffrofêtes} 
\author{Genma}

\begin{document}

%% Titlepage
\begin{frame}
	\titlepage
	\vfill
	\begin{center}
		\CcGroupByNcSa{0.83}{0.95ex}\\[2.5ex]
		{\tiny\CcNote{\CcLongnameByNcSa}}
		\vspace*{-2.5ex}
	\end{center}
\end{frame}

%------------------------------------------------
\begin{frame}
\frametitle{Qu'est ce qu'une chiffofête?}

\begin{block}{Chiffofête, cryptopartie...}
\begin{itemize}
\justifying{
\item Le terme de cryptoparty (contraction de crypto - chiffrement et party - partie, fête) est souvent francisé en crytpopartie mais nous utilisons le terme de chiffofête (contraction de chiffrement et fête) qui se veut une traduction moins connoté de ce terme.
\item  L'autre appellation que nous utilisons pour ce même type d'évènement étant les Café vie privée.
}
\end{itemize}
\end{block}
\end{frame}


%------------------------------------------------
\begin{frame}
\frametitle{Le public concerné}

\begin{block}{Quel est le public ciblé/concerné par les chiffofêtes?}
Le public est divers et varié :
\begin{itemize}
\justifying{
\item Journalistes
\item  Scolaires (Lycéens, étudiants)
\item  Grand public (tout âge confondu)
}
\end{itemize}
\justifying{
D'une façon générale, ce peut être toute personne sensibilisée / concernées par les problématiques de la vie privée, de la sécurisation de ses communications...
}
\end{block}
\end{frame}


%------------------------------------------------
\begin{frame}
\frametitle{Ateliers}

\begin{block}{Quels sont les ateliers possibles?}
\begin{itemize}
\justifying{
\item  OTR “off the record”
\item  TOR : Présentation de Tor, Utilisation via le Tor Browser bundle et/ou le live-cd Tails
\item  TrueCrypt : Présentation, Création d'un conteneur chiffré, Utilisation avancée (Conteneur caché)
\item  GPG : Thunderbird et Enigmail, création et gestion des clefs, réseau de confiance...
\item  Le tracking sur Internet : utilisation d'extensions pour Firefox
\item  Darknet : I2P etc.
\item  Crytpoavancée : chiffrement de serveurs, TAHOELFS...
}
\end{itemize}
\end{block}
\end{frame}

%------------------------------------------------
\begin{frame}
\frametitle{Mini-conférences et conférences}

\begin{block}{Mini-conférences et conférences}
\begin{itemize}
\justifying{
\item  Au cours d'une chiffofête, une ou plusieurs conférences peuvent être présentées au public. 
\item  La présentation d'une conférence peut être indépendante de l'organisation d'une chiffofête et les conférences peuvent être regroupées/fusionnées pour faire une conférence globale abordant différents thèmes au cours d'une après-midi ou sur une journée entière par exemple.
}
\end{itemize} 
\end{block}
\end{frame}

%------------------------------------------------
\begin{frame}
\frametitle{Organiser}

\begin{block}{Qui peut faire une chiffofête?}
\begin{itemize}
\justifying{
\item Toute personne qui a les connaissances minimales et qui souhaite les partager peut se lancer dans la mise en place de sa propre chiffofête.
}
\end{itemize}
\end{block}
\end{frame}

%------------------------------------------------
\begin{frame}
\frametitle{Logisitique}

\begin{block}{Quelle est la logistique d'une chiffofête?} 
\justifying{Les éléments suivants ont été identifiés : }
\begin{itemize}
\justifying{
\item connaitre la capacité d'accueil du lieu, avoir un ou plusieurs contacts référents
\item prévoir une inscription en ligne (pour évaluer le nombre de participants) sur un outil permettant l'anonymat
\item prévoir un accès à Internet via un réseau filaire (câbles ethernet + switch) et/ou Wifi.
\item prévoir chaises, tables, rallonges électriques en quantité suffisante
\item un vidéo-projecteur 
}
\end{itemize}
\end{block}
\end{frame}


%------------------------------------------------
\begin{frame}
\frametitle{Le lieu}

\begin{block}{Quels sont les lieux susceptibles d'accueillir une chiffofête?}
\begin{itemize}
\justifying{
\item   
Les chiffofêtes étant destinées à un public varié, du grand public aux utilisateurs plus avancé, les lieux peuvent être des médiathèques, des salons informatiques...
}
\end{itemize}
\end{block}
\end{frame}


%------------------------------------------------
\begin{frame}
\frametitle{Communication}

\begin{block}{Quelques slogans peut-on utiliser pour promouvoir les chiffofête?}
\begin{itemize}
\justifying{
\item Venez apprendre à protéger vos communications en ligne !
\item Chiffrement et anonymat : reprenez le contrôle de votre vie privée.
\item Surfez et chattez couverts avec des cypherpunks gentils.
}
\end{itemize}
\end{block}

\begin{block}{Ou encore}
\begin{itemize}
\justifying{
\item Parce que préserver sa vie privée est un droit,
\item Parce qu’on peut avoir envie de ne pas être espionné,
\item Parce que l'on a TOUS quelque chose à cacher,
\item Parce que les outils existent et ne sont pas si compliqués…
\item Venez apprendre à vous protéger en ligne !
}
\end{itemize}
\end{block}
\end{frame}


%------------------------------------------------
\begin{frame}
\frametitle{Des conseils}

\begin{block}{Quelques conseils pour le déroulement de la chiffofête?}
\begin{itemize}
\justifying{
\item S'adapter au niveau des participants (du débutant à l'utilisateur avancé)
\item Prendre en compte des besoins et des attentes du public.
\item En début de séance, un petit sondage/tour de table permet de définir les attentes et les ateliers qui seront
\item La durée conseillée pour les ateliers est de 1h30.
\item Deux ateliers successifs semblent suffisant pour commencer (cela fait 3h avec une pause entre les deux).
}
\end{itemize}
\end{block}
\end{frame}
  

%------------------------------------------------
\begin{frame}
\frametitle{Le logiciel libre}

\begin{block}{Et le logiciel libre dans tout ça?}
\begin{itemize}
\justifying{
\item Dès que possible, c'est le logiciel libre qui est privilégié. Chacun vient avec son ordinateur et quelque soit le système d'exploitation (GNU/Linux, MacOSX, Windows, Android), les logiciels les plus adaptés sont proposés, installés.
}
\end{itemize}
\end{block}
\end{frame}


%----------------------------------------------------------------------------------------
\begin{frame}
\Huge{\centerline{Merci de votre attention.}}
\Huge{\centerline{Place aux questions. Débattons...}}
\end{frame}

\end{document}
